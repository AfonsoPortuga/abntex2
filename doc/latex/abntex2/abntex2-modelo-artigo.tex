%% abtex2-modelo-artigo.tex, v<VERSION> laurocesar
%% Copyright 2012-2013 by abnTeX2 group at http://code.google.com/p/abntex2/ 
%%
%% This work may be distributed and/or modified under the
%% conditions of the LaTeX Project Public License, either version 1.3
%% of this license or (at your option) any later version.
%% The latest version of this license is in
%%   http://www.latex-project.org/lppl.txt
%% and version 1.3 or later is part of all distributions of LaTeX
%% version 2005/12/01 or later.
%%
%% This work has the LPPL maintenance status `maintained'.
%% 
%% The Current Maintainer of this work is the abnTeX2 team, led
%% by Lauro César Araujo. Further information are available on 
%% http://code.google.com/p/abntex2/
%%
%% This work consists of the files abntex2-modelo-artigo.tex and
%% abntex2-modelo-references.bib
%%
%%
%% 2013.1.16 19h21	laurocesar
%%  Usa fonte Latin Modern
%%
%% 2013.1.14 21h25 laurocesar
%%  Revisa erros de ortografia 
%%  Altera o título de ``Modelo Canônico de Artigos Acadêmicos'' para ``Modelo
%% Canônico de Artigo Científico''
%%
%% 2013.1.13 09h44	laurocesar
%%  Adiciona seção de explicação do recuo do ambiente citacao
%%  Revisa ortografia
%%
%% 2013.1.12 12h13	laurocesar
%%  Conclusão do modelo
%%
%% 2013.1.7 12h49	laurocesar
%%  Criado o modelo
%%

% ------------------------------------------------------------------------
% ------------------------------------------------------------------------
% Modelo de Artigo Acadêmico utilizando abnTeX2 
% ------------------------------------------------------------------------
% ------------------------------------------------------------------------
%
% oneside = apenas frente
% twocolumn = artigo em duas colunas
\documentclass[article,11pt,oneside,a4paper]{abntex2}	

% ---
% PACOTES
% ---

% ---
% Pacotes fundamentais 
% ---
\usepackage{cmap}				% Mapear caracteres especiais no PDF
\usepackage{lmodern}			% Usa a fonte Latin Modern
\usepackage[T1]{fontenc}		% Seleção de códigos de fonte.
\usepackage[utf8]{inputenc}		% Determina a codificação utiizada (conversão automática dos acentos)
\usepackage{makeidx}            % Cria o indice
\usepackage{hyperref}  			% Controla a formação do índice
\usepackage{indentfirst}		% Indenta o primeiro parágrafo de cada seção.
\usepackage{nomencl} 			% Lista de simbolos
\usepackage{color}				% Controle das cores
\usepackage{graphicx}			% Inclusão de gráficos
% ---
		
% ---
% Pacotes adicionais, usados apenas no âmbito do Modelo Canônico do abnteX2
% ---
\usepackage{lipsum}				% para geração de dummy text
% ---
		
% ---
% Pacotes de citações
% ---
\usepackage[brazilian,hyperpageref]{backref}	 % Paginas com as citações na bibl
\usepackage[alf]{abntex2cite}	% Citações padrão ABNT
% ---

% ---
% Configurações do pacote backref
% Usado sem a opção hyperpageref de backref
\renewcommand{\backrefpagesname}{Citado na(s) página(s):~}
% Texto padrão antes do número das páginas
\renewcommand{\backref}{}
% Define os textos da citação
\renewcommand*{\backrefalt}[4]{
	\ifcase #1 %
		Nenhuma citação no texto.%
	\or
		Citado na página #2.%
	\else
		Citado #1 vezes nas páginas #2.%
	\fi}%
% ---

% ---
% Informações de dados para CAPA e FOLHA DE ROSTO
% ---
\titulo{Modelo Canônico de\\ Artigo Científico com \abnTeX}
\autor{Equipe \abnTeX\thanks{\url{http://code.google.com/p/abntex2/}} \\ Lauro
César
Araujo\thanks{laurocesar@laurocesar.com}}
\local{Brasil}
\data{2013, v<VERSION>}
% ---

% ---
% Configurações de aparência do PDF final

% alterando o aspecto da cor azul
\definecolor{blue}{RGB}{41,5,195}

% informações do PDF
\hypersetup{
     	%backref=true,
     	%pagebackref=true,
		%bookmarks=true,         		% show bookmarks bar?
		pdftitle={\imprimirtitulo}, 
		pdfauthor={\imprimirautor},
    	pdfsubject={\imprimirpreambulo},
		pdfkeywords={PALAVRAS}{CHAVES}{EM}{PORTUGUES},
	    pdfproducer={LaTeX with abnTeX2}, 	% producer of the document
	    pdfcreator={\imprimirautor},
    	colorlinks=true,       		% false: boxed links; true: colored links
    	linkcolor=blue,          	% color of internal links
    	citecolor=blue,        		% color of links to bibliography
    	filecolor=magenta,      		% color of file links
		urlcolor=blue,
		bookmarksdepth=4
}
% --- 

% ---
% compila o indice
% ---
\makeindex
% ---

% ---
% Altera as margens padrões
% ---
\setlrmarginsandblock{4cm}{4cm}{*}
\setulmarginsandblock{4cm}{3.5cm}{*}
\checkandfixthelayout
% ---

% --- 
% Espaçamentos entre linhas e parágrafos 
% --- 

% O tamanho do parágrafo é dado por:
\setlength{\parindent}{1.3cm}

% Controle do espaçamento entre um parágrafo e outro:
\setlength{\parskip}{0.2cm}  % tente também \onelineskip

% ----
% Início do documento
% ----
\begin{document}

% ----------------------------------------------------------
% ELEMENTOS PRÉ-TEXTUAIS
% ----------------------------------------------------------

%---
%
% Se desejar escrever o artigo em duas colunas, descomente a linha abaixo
% e a linha com o texto ``FIM DE ARTIGO EM DUAS COLUNAS''.
% \twocolumn[    		% INICIO DE ARTIGO EM DUAS COLUNAS
%
%---
% página de titulo
\maketitle

% resumo em português
\begin{resumoumacoluna}
 Conforme a ABNT NBR 6022:2003, o resumo é elemento obrigatório, constituído de
 uma sequência de frases concisas e objetivas e não de uma simples enumeração
 de tópicos, não ultrapassando 250 palavras, seguido, logo abaixo, das palavras
 representativas do conteúdo do trabalho, isto é, palavras-chave e/ou
 descritores, conforme a NBR 6028. (\ldots) As palavras-chave devem figurar logo
 abaixo do resumo, antecedidas da expressão Palavras-chave:, separadas entre si por
 ponto e finalizadas também por ponto.
 
 \vspace{\onelineskip}
 
 \noindent
 \textbf{Palavras-chaves}: latex. abntex. editoração de texto.
\end{resumoumacoluna}

% ]  				% FIM DE ARTIGO EM DUAS COLUNAS
% ---

% ----------------------------------------------------------
% ELEMENTOS TEXTUAIS
% ----------------------------------------------------------
% É possível usar \textual ou \mainmatter, que é a macro padrão do memoir.  
\mainmatter

% ----------------------------------------------------------
% Introdução
% ----------------------------------------------------------
\section*{Introdução}

Este documento exemplifica o uso da classe \textsf{abntex2} para elaboração de
artigos em publicação periódica científica impressa produzidos conforme a ABNT
NBR 6022:2003 \emph{Informação e documentação - Artigo em publicação periódica
científica impressa - Apresentação}.

A expressão ``Modelo canônico'' é utilizada para indicar que \abnTeX~não é
modelo específico de trabalho acadêmico de nenhuma universidade ou instituição,
mas que implementa tão somente os requisitos das normas da ABNT.

Sinta-se convidado a participar do projeto \abnTeX! Acesse o site do projeto em
\url{http://code.google.com/p/abntex2/}. Também fique livre para conhecer,
estudar, alterar e redistribuir o trabalho do \abnTeX, desde que os arquivos
modificados tenham seus nomes alterados e que os créditos sejam dados aos
autores originais, nos termos da ``The LaTeX Project Public
License''\footnote{\url{http://www.latex-project.org/lppl.txt}}.

Encorajamos que sejam realizadas customizações específicas deste documento.
Porém, recomendamos que ao invés de se alterar diretamente os arquivos do
\abnTeX, distribua-se arquivos com as respectivas customizações. Isso permite
que futuras versões do \abnTeX~não se tornem automaticamente incompatíveis com
as customizações promovidas.

Este exemplo deve ser utilizado como complemento do manual da classe
\textsf{abntex2} \cite{abntex2classe}, dos manuais do pacote
\textsf{abntex2cite} \cite{abntex2cite,abntex2cite-alf} e do manual da classe
\textsf{memoir} \cite{memoir}. Consulte o \citeonline{abntex2modelo} para obter
exemplos e informações adicionais de uso de \abnTeX\ e de \LaTeX.

% ----------------------------------------------------------
% Seção de explicações
% ----------------------------------------------------------
\section{Exemplos de comandos}

\subsection{Margens}

A norma ABNT NBR 6022:2003 não estabelece uma margem específica a ser utilizada
no artigo científico. Dessa maneira, caso desej alterar as margens, utilize os
comandos abaixo:

\begin{verbatim}
   \setlrmarginsandblock{4cm}{4cm}{*}
   \setulmarginsandblock{4cm}{3.5cm}{*}
   \checkandfixthelayout
\end{verbatim}

\section{Duas colunas}

É comum que artigos científicos sejam escritos em duas colunas. Para isso,
adicione a opção \texttt{twocolumn} à classe do documento, como no exemplo:

\begin{verbatim}
   \documentclass[article,11pt,oneside,a4paper,twocolumn]{abntex2}
\end{verbatim}

É possível indicar pontos do texto que se deseja manter em apenas uma coluna,
geralmente o título e os resumos. Os resumos em única coluna em documentos com
a opção \texttt{twocolumn} devem ser escritos no ambiente
\texttt{resumoumacoluna}:

\begin{verbatim}
   \twocolumn[              % INICIO DE ARTIGO EM DUAS COLUNAS

     \maketitle             % pagina de titulo

     \renewcommand{\resumoname}{Nome do resumo}
     \begin{resumoumacoluna}
        Texto do resumo.
      
        \vspace{\onelineskip}
 
        \noindent
        \textbf{Palavras-chaves}: latex. abntex. editoração de texto.
     \end{resumoumacoluna}
   
   ]                        % FIM DE ARTIGO EM DUAS COLUNAS
\end{verbatim}

\section{Recuso do ambiente \texttt{citacao}}

No produção de artigos (opção \texttt{article}), pode ser útil alterar o recuo
do ambiente \texttt{citacao}. Nesse caso, utilize o comando:

\begin{verbatim}
   \setlength{\ABNTEXcitacaorecuo}{1.8cm}
\end{verbatim}

Quando um documento é produzido com a opção \texttt{twocolumn}, a classe
\textsf{abntex2} automaticamente altera o recuo padrão de 4 cm, definido pela
ABNT NBR 10520:2002 seção 5.3, para 1.8cm.

\subsection{Consulte o Modelo Canônico de Trabalhos Acadêmicos}

Este modelo de artigo é limitado em número de exemplos de comandos. São
apresentados exclusivamente comandos diretamente relacionados com a produção de
artigos.

Para exemplos adicionais de \abnTeX2 e \LaTeX, como inclusão de figuras,
fórmulas matemáticas, citações, e outros, consulte o documento ``Modelo Canônico
de Trabalhos Acadêmicos com abnTeX2'' \cite{abntex2modelo}.

\subsection{Consulte o manual da classe \textsf{abntex2}}

Consulte o manual da classe \textsf{abntex2} \cite{abntex2classe} para uma
referência completa das macros e ambientes disponíveis.

% ---
% Finaliza a parte no bookmark do PDF, para que se inicie o bookmark na raiz
% ---
\bookmarksetup{startatroot}% 
% ---

% ----------------------------------------------------------
% ELEMENTOS PÓS-TEXTUAIS
% ----------------------------------------------------------
\postextual

% ---
% Conclusão
% ---
\section*{Conclusão}
\addcontentsline{toc}{section}{Conclusão}

\lipsum[1]

\begin{citacao}
\lipsum[2]
\end{citacao}

\lipsum[3]

% ----------------------------------------------------------
% Referências bibliográficas
% ----------------------------------------------------------
\bibliography{abntex2-modelo-references}

% ----------------------------------------------------------
% Glossário
% ----------------------------------------------------------
%
% Há diversas soluções prontas para glossário. Não é necessário nos preocuparmos
% com isso agora.
%
%\glossary

% ----------------------------------------------------------
% Apêndices
% ----------------------------------------------------------

% ---
% Inicia os apêndices
% ---
\begin{apendicesenv}

% ----------------------------------------------------------
\chapter{Nullam elementum urna vel imperdiet sodales elit ipsum pharetra ligula
ac pretium ante justo a nulla curabitur tristique arcu eu metus}
% ----------------------------------------------------------
\lipsum[55-57]

\end{apendicesenv}
% ---

% ----------------------------------------------------------
% Anexos
% ----------------------------------------------------------
\cftinserthook{toc}{AAA}
% ---
% Inicia os anexos
% ---
%\anexos
\begin{anexosenv}

% ---
\chapter{Cras non urna sed feugiat cum sociis natoque penatibus et magnis dis
parturient montes nascetur ridiculus mus}
% ---

\lipsum[31]

\end{anexosenv}


% ---
% Título e resumo em língua estrangeira
% ---

% \twocolumn[    		% INICIO DE ARTIGO EM DUAS COLUNAS

% titulo em inglês
\titulo{Canonical academic article model with \abnTeX}
\maketitle

% resumo em português
\renewcommand{\resumoname}{Abstract}
\begin{resumoumacoluna}
 According to ABNT NBR 6022:2003, an abstract in foreign language is a back
 matter mandatory element.

 \vspace{\onelineskip}
 
 \noindent
 \textbf{Key-words}: latex. abntex.
\end{resumoumacoluna}

% ]  				% FIM DE ARTIGO EM DUAS COLUNAS
% ---

\end{document}